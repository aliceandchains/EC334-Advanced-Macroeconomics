\documentclass{article}
\usepackage{graphicx}
\usepackage{booktabs}
\usepackage{multirow}
\usepackage{adjustbox}
\usepackage[a4paper, margin=1in]{geometry}
\usepackage{graphicx}
\usepackage{caption}
\usepackage{subcaption}
\usepackage{float}  % Optional, for [H] positioning

\title{Did the Phillips Curve Fail After the Pandemic? Evidence from U.S. Inflation}
\author{Advanced Macroeconomics (EC334)}
\date{Jan 2026}

\begin{document}

\maketitle

\section{Abstract}

The Phillips Curve plays a central role in modern macroeconomic models and monetary policy analysis, yet it appeared to perform poorly in predicting inflation in the United States following the Covid-19 pandemic. This paper examines whether this episode reflects a structural breakdown of the inflation–slack relationship or a failure of aggregate forecasting frameworks. Using expectations-augmented Phillips Curve specifications, the analysis compares pre- and post-pandemic inflation dynamics and decomposes inflation into components driven by expectations, labour market conditions, and residual factors. The results suggest that post-pandemic inflation was largely driven by shifts in inflation expectations and broad supply-side forces, rather than labour market slack, implying a breakdown in forecasting performance rather than a collapse of the underlying Phillips Curve relationship.
 
\addlinespace




\end{document}
